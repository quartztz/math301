\chapter{Jack gets pedantic about notation for like half a page}

Now. Professor Lachowska is a doctor in mathematics, and also potentially a good dozen of orders of magnitudes smarter than me. So this is by no means a way for me to claim I'm better than her. However. 

When you say that $\mathbb{R}[x]$ is the set of all polynomials of one variable over coefficients in $\mathbb{R}$, what you're doing in computer science terms is you're defining a type. An element of that set essentially follows the rules of that type. The way the type $\mathbb{R}[x]$ is typically notated is $\mathbb{R} \to \mathbb{R}$, and we define an element of that type as $f: \mathbb{R} \to \mathbb{R}$. 

$f(x)$ means the application of $f: \mathbb{R} \to \mathbb{R}$ to the number $x: \mathbb{R}$, and it is therefore a real value, $f(x): \mathbb{R}$. 

What Prof. Lachowska says is: ``Nah, I'll do my own thing''. $f(x)$ is the polynomial, and $f$ does not exist. This makes notation heavier and not standard. 

Why does she do that? I'm not sure. Does it matter? Not really. Am I bitter about it? Yes. 

And it's my textbook, so I get to write this page. 
\chapter{Groups}
\marginpar{Week 2}

\section{Base definitions}

\begin{definition}
  A \emph{group} is a set $G$ with a binary operation $\cdot: G \times G \to G$, satisfying the following axioms: 
  \begin{itemize}
    \item $\cdot$ is \emph{associative}: $(a \cdot b) \cdot c = a \cdot (b \cdot c)$
    \item There exists a \emph{neutral element} $e$ such that $a \cdot e = e \cdot a = a \ \forall a \in G$. 
    \item For any $a \in G$ there exists an \emph{inverse} $a^{-1}$ such that $a^{-1} \cdot a = a \cdot a^{-1} = e$. 
  \end{itemize}
  We say that $G$ is a \emph{finite} group if $|G| < \infty$. In that case, we say that $G$ is of \emph{order} $|G|$. 
  We say that $G$ is \emph{abelian} (or commutative) if $a\cdot b = b\cdot a \ \forall a, b \in G$. 
\end{definition}
\begin{definition}
  $H \subset G$ is a \emph{subgroup} if it contains the neutral element $e_G$ and if it is closed with respect to $\cdot_G$, meaning that for every $a, b \in H$, $a\cdot b \in H$, and to inverses. 
\end{definition}

We can note that any group has a subgroup generated by a single element:
\[
  \langle g \rangle = \{e, g^{1}, g^{2}, \dots, g^{-1}, g^{-2}, \dots\}
\]
Since $g^i \cdot g^j = g^{i + j}$ by definition of the group operation, this set is closed under it, meaning it is a subgroup. 

\begin{definition}
  If it exists, the minimal $n \in \mathbb{N}^*$ such that $g^n = e$ is called the \emph{order} of $g$. It is finite for every element in a finite group. 
\end{definition}

\section{Cosets}

\begin{definition}
  Let $H\subset G$ be a subgroup of G. The \emph{left coset} of $g$ with respect to $H$, denoted $gH$, is the following set: 
  \[
    gH = \{gh, h \in H\}
  \]
\end{definition}

\begin{theorem}
  Let $H \subset G$ finite. Then: 
  \begin{enumerate}
    \item Two left-cosets $xH, yH$ are either disjoint ($xH \cap yH = \emptyset$) or equal. 
    \item For any element $g \in G$ there exists a left coset of $H$ such that $g \in H$.
    \item $|xH| = |H| \ \forall x \in G$
  \end{enumerate}
\end{theorem}
\begin{proof}
  We will prove each part separately: 
  \begin{enumerate}
    \item Suppose $xH, yH$ are such that $xH \cap yH \neq \emptyset$. This means that there exist $h_1, h_2$ such that $xh_1 = yh_2$. Therefore, 
    \[
      x = yh_2h_1^{-1} = yh_3 \in yH \Rightarrow xh = yh_3h \ \forall h \in H
    \]
    This means that if there exists an element of $xH$ that is in $yH$, then every element in $xH$ can be written as an element in $yH$, meaning they are equal.
    \item For any $g \in G$, one can construct $gH = \{e, g, g^2, ...\}$, which naturally contains $g$. 
    \item The mapping 
    \begin{align*}
      f(h): H     &\to xH \\
            h &\mapsto xh
    \end{align*}
    is surjective, by definition of $xH = \{xh, h \in H\}$, and it is also injective, since $xh_1 = yh_2 \Leftrightarrow h_1 = h_2$. This means it defines a bijection between $H$ and $xH$, indicating they have the same cardinality.
  \end{enumerate}
  \aparte{Example}{
    Let $G = (\mathbb{Z}, +, 0), H = 3\mathbb{Z}\subset \mathbb{Z}$. The left coset of 0 with respect to $H$ is : 
    \[
      \{0 + 3k\}_{k \in \mathbb{Z}} = H = \{3 + 3k\}_{k \in \mathbb{Z}}
    \]
    The left coset of 1 is 
    \[
      \{1 + 3k\}_{k \in \mathbb{Z}} = \{1, 4, 7, -2, ...\}
    \]
  }
\end{proof}

\begin{theorem}[Lagrange]
  Let $G$ be a finite group, $H \subset G$ a subgroup. Then, $|H|$ divides $|G|$.
\end{theorem}
\begin{proof}
  Each $g \in G$ belongs to a left coset of $H$, which are either disjoint or equal. This means:
  \begin{align*}
    G &= \bigcup_{i = 0}^{r}x_iH &[\text{disjoint union of finitely many sets}] \\
    \Rightarrow |G| &= \sum_{i = 0}^r |x_iH| &\\
    \Rightarrow |G| &= \sum_{i = 0}^r |H| &[\text{since } |xH| = |H|] \\
    \Rightarrow |G| &= r|H| &
  \end{align*}
  with $r \in \mathbb{N}$, meaning that $|H|$ divides $|G|$. 
\end{proof}

\begin{definition}
  The number of left cosets of $H$ of $G$ is called the \emph{index} of $G$: 
  \[
    [G : H] = |G|/|H| \in \mathbb{N}^*
  \]
\end{definition}

This means that the order of any element $g \in G$ (notated $\operatorname{ord}(g)$) divides the order of the group $|G|$, since every element generates a subgroup $\langle g \rangle$. Additionally, it implies 

\begin{corollary}\label{stupid}
  $g^{|G|} = (g^{\operatorname{ord}(g)})^{k} = e^k = e$ for some $k$.
\end{corollary}

\section{RSA}

\begin{theorem}[Euler's theorem]
  Let $a, n \in \mathbb{Z}^+$. such that $\gcd(a, n) = 1$. Then, \[ a^{\varphi(n)} \equiv 1 \ \operatorname{mod} n\]
\end{theorem}
\begin{proof}
  Consider $G = (\mathbb{Z}/n\mathbb{Z}, \cdot, 1)$. Then, 
  \[
    a^{\varphi(n)} = a^{|G|} \stackrel{\ref{stupid}}{=} 1
  \]
\end{proof}

\begin{theorem}[Fermat's little theorem]
  Let $a \in \mathbb{Z}^+$, $p$ prime such that $p$ does not divide $a$. Then, $a^{p - 1} = 1$. 
\end{theorem}
\begin{proof}
  Consider $G = (\mathbb{Z}/p\mathbb{Z}, \cdot, 1)$. Then, $|G| = \varphi(p) = p - 1$. By Euler's theorem, \[a^{\varphi(p)} = a^{(p - 1)} = 1\]
\end{proof}

\aparte{RSA}{
  The RSA cryptosystem for message transmission works as follows: 
  \begin{enumerate}
    \item Choose two distinct large primes $p, q$. 
    \item Compute $m = pq \Rightarrow \varphi(m) = (p - 1)(q - 1)$. 
    \item Choose $e \leqslant m$ an encryption key such that $\gcd(e, \varphi(m)) = 1$. 
    \item Use Euclid's algorithm to determine $d$ such that $ed - k\varphi(m) = 1$ for some integer $k$.
    \item The encoding key is the pair $(m, e)$, and it can be published. To decode, you use the decoding key $(m, d)$ which is to be kept private. 
  \end{enumerate}

  To send a message $x$ to someone, you need their public pair $(m, e)$. You first compute $c \equiv x^e \operatorname{mod} m$, which can be sent publicly. To decode, the person will use their private pair $(m, d)$, computing $x \equiv c^d \operatorname{mod} m \equiv x^{ed} \operatorname{mod} m$.
}

Why is it the case that $x^{ed} \equiv x \operatorname{mod} m$? Well...

\begin{theorem}
  Let $p, q$ be two distinct primes, and $m = pq$. Let $e: \gcd(e, \varphi(m)) = 1$, and let $d \in \mathbb{Z}: ed - k\varphi(m) = 1$ for some $k \in \mathbb{Z}$. Then, 
  \[
    x^{ed} \equiv x \operatorname{mod} m
  \]
  for all $x \in \{1, ..., m\}$.
\end{theorem}

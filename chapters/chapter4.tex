\chapter{Rings and fields}

\marginpar{Week 8}
\section{F*!king finally!}

\begin{definition}
  A \emph{ring} is a set $A$ with two operations defined on it: $+$ and $\cdot$, in such a way that
  \begin{enumerate}
    \item $A$ is an abelian group with respect to $+$, containing a neutral element $0$. 
    \item $\cdot$ is associative: $(a \cdot b) \cdot c = a \cdot (b \cdot c)$, et il $\exists 1 \in A: 1 \cdot a = a \cdot 1$.
    \item  $\cdot$ is both left- and right-distributive over $+$: $a \cdot (b + c) = ab + ac,  (a + b) \cdot c = ac + bc$. 
  \end{enumerate}
\end{definition}
\aparte{Examples}{
  Let us consider some examples: 
  \begin{itemize}
    \item $\mathbb{Z}: \gen{+, \cdot, 0, 1}$ is a ring! Though it doesn't contain any multiplicative inverses, it fits the definitions nicely. 
    \item $A = \mathbb{Z}[\sqrt{2}] := \{a + b\sqrt{2}, a, b \in \mathbb{Z}\}$ contains 0 and 1. Additionally: 
    \begin{gather*}
      (a + b\sqrt{2}) + (c + d \sqrt{2}) = (a + c) + (b + d)\sqrt{2} \in A\\
      -a - b\sqrt{2} \in A \\
      (a + b\sqrt{2})(c + d\sqrt{2}) = ac + 2bd + \sqrt{2}(ad + bc) \in A \\
    \end{gather*}
    \item $\mathbb{Q}[\sqrt{2}]$ is a ring. Proof of this is left as an exercice to the reader. 
  \end{itemize}
}

In this course, we will consider only \emph{commutative rings}, such that $ab = ba \forall a, b \in A$. 

\begin{definition}\label{zerodiv-def}
  $a \in A$ is a \emph{zero divisor} if there exists an $0 \neq x \in A$ such that $ax = 0$. 
\end{definition}
Most rings we are used to do not have any nontrivial zero divisors. Let us consider another one: 
\aparte{Example}{
  Consider: 
  \[
    \mathbb{Z}/n\mathbb{Z} = \{[0], [1], ..., [n - 1]\}
  \]
  It is isomorphic to the cyclic group of order $n$, meaning it is an abelian group with respect to addition. $\cdot$ is associative, and $[1]$ is its neutral element wrt multiplication. 

  Consider now an element $[a] \in \mathbb{Z}/n\mathbb{Z}$. 
  \begin{itemize}
    \item If $\gcd(a, n) = d > 1$, then we can write \[[a] \cdot \left[\frac{n}{d}\right] = [1]\] meaning that $a$ is a nontrivial zero divisor of the ring. 
    \item If $a, n$ are coprime, then by Bezout's theorem 
    \begin{gather*}
      \exists x, y : ax + ny = 1 \Rightarrow [a][x] = [1] \Rightarrow [a]^{-1} = [x]
    \end{gather*}
    Additionally, if we consider $[b]: [b] \cdot [a] = 0$, then we have
    \[
      [b] \cdot [a] = [0] \Rightarrow [b] \cdot [a] \cdot [x] = [0] \Rightarrow [b] \cdot [1] = [0] \Rightarrow [b] = [0]
    \]
    Meaning that $[a]$ cannot be a zero divisor of the ring. 
  \end{itemize}
  This means that an element of $\mathbb{Z}/n\mathbb{Z}$ is either invertible, or a zero divisor. 
}

\begin{definition}
  A ring that has no nontrivial zero divisors is called an \emph{integral domain}.
\end{definition}
\begin{definition}
  A commutative ring where all nonzero elements have a multiplicative inverse is called a \emph{field}. $\forall 0 \neq a \in A \exists a^{-1} \in A : a a^{-1} = a^{-1} a = 1$
\end{definition}
\begin{corollary}
  $\mathbb{Z}/n\mathbb{Z}$ either has nontrivial zero divisors ($\Leftrightarrow n$ is not prime) or it is a field and an integral domain ($\Leftrightarrow n = p$ for some prime $p$). 
\end{corollary}
\begin{proof}
  If it has no nontrivial zero divisors, that means that there is no $[a] \in \{[1], ..., [n - 1]\}$ such that $\gcd(a, n) > 1$. Therefore, $n$ has no divisors other than itself and 1, and it must be a prime $p$ by definition. If that's the case, then $\forall [b] \neq 0, \gcd(b, p) = 1 \Rightarrow [b]$ has a multiplicative inverse, meaning that $\mathbb{Z}/n\mathbb{Z}$ is a field.
\end{proof}

\begin{theorem*}
  A field is an integral domain. An invertible element is not a zero divisor. 
\end{theorem*}
\begin{proof}
  Suppose $ab = 0$ in a field $A$ such that $a \neq 0$. This means that there exists a $a^{-1} \in A$ such that $aa^{-1} = 1$. This means that :
  \begin{align}
    ab = 0 
    &\Leftrightarrow a^{-1}ab = a^{-1} \cdot 0 \\
    &\Leftrightarrow 1 \cdot b = 0 \\
    &\Leftrightarrow b = 0
  \end{align}
  meaning $a$ cannot be a zero divisor. This means that if all non-zero elements in $A$ are invertible, then $A$ cannot have nontrivial zero divisors. This is equivalent to saying that if $A$ is a field, then it is an integral domain, proving our proposition.
\end{proof}
\aparte{Remark}{The converse isn't true! $\mathbb{Z}$ is an integral domain, but $2 \in \mathbb{Z}$ does not have a multiplicative inverse.}

This allows the following characterization: 
\[
  \text{Fields} \subset \text{Integral Domains} \subset \text{Commutative Rings}
\]
And for the following statements on $\mathbb{Z}/n\mathbb{Z}$ : 
\[
  \mathbb{Z}/n\mathbb{Z} \text{ integral domain} \Leftrightarrow \mathbb{Z}/n\mathbb{Z} \text{ field} \Leftrightarrow n = p \text{ prime}
\]

\section{Less than ideal}

\begin{definition}
  Let $A$ be a commutative ring. $I \subset A$ is an \emph{ideal} if it has the following properties:
  \begin{itemize}
    \item it is a subgroup with respect to +, i.e: it contains 0, $-a$ for any $a \in I$, and it is closed wrt +. 
    \item $\forall x \in A, a \in I: x \cdot a \in I$. 
  \end{itemize}
\end{definition}
\aparte{Examples}{
  Let's have a look at some examples
  \begin{itemize}
    \item $\{0\} \subset A, A \subset A$ are ideals for any commutative ring $A$. That is not very interesting.
    \item Consider $\mathbb{Z}$. Then, $2\mathbb{Z} = \{2a, a \in \mathbb{Z}\}$ is an ideal! It's easy to verify it is a subgroup for $+$, and we have that, for any $x \in \mathbb{Z}, 2a \cdot x \in 2\mathbb{Z}$. This is true for any $d \in \mathbb{Z}$. 
  \end{itemize}
}

\begin{definition}
  We say that an ideal $I \subset A$ is \emph{proper} if $I \neq A$, and \emph {nontrivial} if $I \neq 0$. 
\end{definition}
\aparte{Properties}{
  Let $I, J \subset A$ two ideals. Then: 
  \begin{enumerate}
    \item[(i)] if $1 \in I$, then $I = A$. 
    
    \textit{This is due to the properties of ideals: $\forall x \in A, 1 \cdot x \in I \Rightarrow x \in I \Rightarrow I = A$}
    
    \item[(ii)] $I \cap J$ is an ideal

    \textit{Again, easy to verify applying properties of ideals: let $x, y \in I \cap J$.}
    \begin{gather*}
      x + y \in I, J \Rightarrow x + y \in I \cap J \\
      -x \in I, J \Rightarrow -x \in I \cap J \\
      0 \in I, J \Rightarrow 0 \in I \cap J
    \end{gather*}
    \textit{Meaning it is a subgroup. Also, for any $a \in A$, we have that $ax \in I, ax \in J$, which means $ax \in I \cap J$. This makes it an ideal!}

    \item[(iii)] $I + J = \{i + j\}_{i \in I, j \in J}$ is an ideal.

    \item[(iv)] $I \cdot J = \{\sum_{i = 1}^k x_i y_i\}_{x_i \in I, y_i \in J}$ is an ideal.\footnotemark

    \item[(v)] $I \cup J$ is not necessarily an ideal

    \textit{More often than not, it is not an additive subgroup: consider $I = 2\mathbb{Z}$, $J = 3\mathbb{Z}$}
  \end{enumerate}
}
\footnotetext{Proofs to (iii) and (iv) are left as exercice to the reader, as they are similar to that of (ii).}

We can verify some fun properties on integer ideals. Let $A = \mathbb{Z}, I = 6\mathbb{Z}, J = 10\mathbb{Z}$. Then: 
\begin{itemize}
  \item \textbf{Intersection:} We have: 
  \begin{align*}
    I \cap J 
    &= \{z \in \mathbb{Z}: z = 6n \land z = 10m, n, m \in \mathbb{Z}\} \\
    &= \{\text{common multiples of 6 and 10}\} \\
    &= \{\text{all multiples of} \lcm(6, 10) = 30\} \\
    &= 30\mathbb{Z}
  \end{align*}
  \item \textbf{Addition:} We have: 
  \begin{align*}
    I + J 
    &= \{6n + 10m\}_{n, m \in \mathbb{Z}} \\
    &= \{\gcd(6, 10)k, k \in \mathbb{Z}\} \tag*{\text{[By Bezout]}} \\
    &= 2\mathbb{Z}
  \end{align*}
  \item \textbf{Product:} We have: 
  \begin{align*}
    I \cdot J 
    &= \left\{\sum_{i = 1}^k 6x_i \cdot 10y_i, x_i, y_i \in \mathbb{Z}\right\} \\
    &= \left\{ 60\sum_{i = 1}^k x_iy_i,  x_i, y_i \in \mathbb{Z} \right\} \\
    &= 60\mathbb{Z}
  \end{align*}
\end{itemize}

From which we can conclude!

\ 

\begin{unnumthm}
  Let $I = n\mathbb{Z}, J = m\mathbb{Z}$ two ideals of $\mathbb{Z}$. Then: 
  \begin{gather*}
    I \cap J = \lcm(n, m)\mathbb{Z} \\
    I + J = \gcd(n, m)\mathbb{Z} \\
    I \cdot J = (n \cdot m)\mathbb{Z}
  \end{gather*}
\end{unnumthm}

\aparte{Example}{
  Consider $A = \mathbb{R}[x]$ all polynomials on one variable with real coefficients\footnotemark. This forms a ring as is trivial to verify. Then, we consider the following: 
  \begin{gather*}
    I = \{(x + 5)f(x), f(x) \in \mathbb{R}[x]\} \\
    J = \{(x^2 + 2)f(x), f(x) \in \mathbb{R}[x]\}
  \end{gather*}
  which are comprised of all polynomials divisible by $(x + 5)$ and $(x^2 + 2)$, respectively. Then, we have:
  \begin{gather*}
    I \cap J = \{(x + 5)(x^2 + 2)f(x), f(x) \in \mathbb{R}[x]\} \\
    I \cdot J = \left\{\sum_{i = 0}^k (x + 5)f_i(x) \cdot (x^2 + 2)g_i(x)\right\} = \{(x + 5)(x^2 + 2)f(x)\} \\
    I + J = \{(x + 5) f(x) + (x^2 + 2) g(x)\}
  \end{gather*}
  We can find $f(x), g(x)$ such that $(x + 5) f(x) + (x^2 + 2) g(x) = 1 \in \mathbb{R}[x]$ (constant polynomial of value 1). For example, take 
  \[
    (x + 5)(x - 5)\left(\frac{-1}{27}\right) - (x^2 + 2)\left(\frac{-1}{27}\right) = (x^2 - 25 - x^2 - 2)\left(\frac{-1}{27}\right) = 1
  \]

  Can we do that for any pair of polynomials in $\mathbb{R}[x]$?
}
\footnotetext{If you want to see me get pedantic about notation for life half a page, go look at appendix C.}

\marginpar{Week 9}

\begin{definition}\label{princid}
  Let $S \subset A$. Let $I$ be the minimal ideal containing $S$. Then, we denote $I = (S)$ the \emph{ideal generated by the set} $S$. It means: 
  \[
    (S) = \left\{\sum_i s_i a\right\}_{s_i \in S, a \in A}
  \] 
  $I$ is called \emph{principal} if $I = (x) = \{xa, x \in I\}_{a \in A}$ is generated by a single element. 
\end{definition}
\aparte{Examples}{
  $A$ and $\{0\}$ are principal, being generated by $1$ and $0$ respectively. Additionally, $n\mathbb{Z} = (n)$ is principal. 
}

\begin{theorem*}
  $A$ is a field $\Leftrightarrow$ $A, \{0\}$ are the only ideals in $A$
\end{theorem*}
\begin{proof}
  We prove both directions separately: 
  \begin{itemize}
    \item[$\Rightarrow$] $A$ is a field, consider an ideal $I$ that isn't $\{0\}$ and an element $a \in I$, which is not 0. Since we're in a field, and it is non-zero, there exists $a^{-1} \in A$. By definition of an ideal, we have that $a a^{-1} \in I \Rightarrow 1 \in I \Rightarrow I = A$. 
    \item[$\Leftarrow$] Let $A$ be a ring such that $A$, $\{0\}$ are its only ideals. Therefore, we consider an $0 \neq a \in A$, and the ideal it generates. Since $a \neq 0$,we know that it must generate $A$, and that there must exist a $y \in A$ such that $ya = 1$, meaning that $y = a^{-1}$ by definition, and $A$ is a field as a consequence of its existence.  
  \end{itemize} 
\end{proof}

\section{Quotient rings}

\begin{definition}
  An \emph{equivalence relation} $\sim$ on a set $E$ is a relation such that for any $a, b, c \in E$, we have: 
  \begin{itemize}
    \item $a \sim a$, meaning it is \emph{reflexive}
    \item $a \sim b \Leftrightarrow b \sim a$, meaning it is \emph{symmetric}
    \item $a \sim b, b \sim c \Rightarrow a \sim c$, meaning it is \emph{transitive}
  \end{itemize}

  A \emph{congruence relation} $\sim$ on a commutative ring $A$ is an equivalence relation such that for any $a, b, c, d \in A$
  \[
    \begin{cases}
      a \sim b \\
      c \sim d
    \end{cases} \Rightarrow 
    \begin{cases}
      a + c \sim b + d \\
      ac \sim bd
    \end{cases}
  \]
\end{definition}

\begin{theorem*}
  If $I \subset A$ is an ideal, then the relation : 
  \[
    a \sim b \Leftrightarrow (b - a) \in I
  \]
  is an equivalence relation, and if it is a congruence relation, then the set $I = \{a \in A: a \sim 0\}$ is an ideal in $A$. 
\end{theorem*}
\begin{proof}
  We first check all of the properties of an equivalence relation. We have: 
  \begin{itemize}
    \item Any ideal contains the 0 element, meaning that $(a - a) \in I \Rightarrow a \sim a$. This makes $\sim$ reflexive. 
    \item $I$ must be an additive subgroup of $A$. Therefore, it must contain $b - a$ and $-(b - a) = a - b$, meaning that if $a \sim b$, then $b \sim a$. $\sim$ is therefore symmetric. 
    \item Consider $a, b, c \in A$ such that $a \sim b, b \sim c$. Therefore, this means that $c - b \in I, b - a \in I$. Now, consider that $c - a = (c - b) + (b - a)$ is the sum of two elements of $I$: since $I$ is an additive subgroup, that must mean that $c - a \in I$, meaning $a \sim c$. Therefore $\sim$ is transitive. 
  \end{itemize}
  Now that we have checked the properties of an equivalence relation, we must check those for a congruence relationship: consider $a \sim b, c \sim d$ in $A$. Then: 
  \[
    \begin{cases}
      (b + d) - (a + c) = (b - a) + (d - c) \in I \\
      bd - ac = (b - a)(d - c) = e(b - a) \in I
    \end{cases}
  \]
  Where the last step is motivated by the fact that $e = d - c$ is an element of $A$, and by the definition of an ideal. This concludes the proof of the first part of the theorem. 
  
  Then, consider $a \sim 0, b \sim 0$. We have: $a + b \sim 0, 0 \sim 0, -a \sim 0$, which completes the subgroup requirement, as well as $x \sim x \Rightarrow a \cdot x \sim 0 \cdot x \Rightarrow a \cdot x \sim 0 \Rightarrow \{a \in A: a \sim 0\}$ is an ideal! 
\end{proof}

\begin{theorem*}
  Let $A$ be a commutative ring, $\sim$ a congruence relation over $A$ such that $1 \not\sim 0$. Then, the set of congruence classes $A/\{x \in A : x \sim 0\} := A/{\sim}$ is a commutative ring. 
\end{theorem*}
\begin{proof}
  Notate $\bar{a} = \{x \in A: x \sim a\}$. Define $\bar{a} + \bar{b} = \bar{a + b}$, $\bar{a} \cdot \bar{b} = \bar{ab}$. These are well-defined thanks to the properties on $\sim$. Most importantly, $\bar{1} \in A/{\sim}$. 
\end{proof}

\aparte{Example}{
  Consider the ring of polynomials with real coefficients $\mathbb{R}[x]$. Let $I = \langle (x^2 - 4) \rangle$, and consider the commutative ring $B = \mathbb{R}[x]/I$. Within it, we have: 
  \[
    \bar{(x + 2)} \cdot \bar{(x + 1)} = \bar{x^2 + 3x + 2} = {\color{redish} \bar{x^2 + 3x + 2} - \bar{x^2 - 4}} = \bar{3x + 6}
  \]
  where the step in red is justified as a way to find the remainder of the function by the defined congruence relation. Another example: 
  \[
    \bar{x} \cdot \bar{x} = \bar{x^2} \mathbin{\color{redish}=} \bar{4}
  \]
  through a similar process as earlier. One can also look for the zero divisors (recall definition \ref{zerodiv-def}) within this ring. Consider: 
  \[
    \bar{(x + 2)}\bar{(x - 2)} = \bar{(x^2 - 4)} \mathbin{\color{redish}=} \bar{0}
  \]
  since $(x + 2), (x - 2)$ are non-zero, but their product is, this means that they are zero divisors, and that $B$ is not an integral domain. 
  
  As an exercice, one can show that any element in $B$ can be written under the form $\bar{ax + b}, a, b \in \mathbb{R}$. 
}

\begin{definition}
  Recall the definition of a principal ideal (Definition \ref{princid}). A commutative ring where every ideal is principal is called a \emph{principal ring}. An integral domain where every ideal is principal is called a \emph{principal integral domain} (PID). 
\end{definition}

This means that any field is a PID, since it only has two ideals, and they are both principal ($A$ is generated by 1, ${0}$ is generated by 0).

\begin{theorem*}
  $\mathbb{Z}$ is a PID. 
\end{theorem*}
\begin{proof}
  If $I = \{0\}$, then $I = (0)$. Suppose then that $I \neq \{0\}$. Therefore, there exists an $0 \neq a \in I$, such that $a \in I, -a \in I \Rightarrow |a| \in I$. Consider $d$ the smallest positive element in $I$. Then, for any $n \in I$, by euclidean division, $n = kd + r, 0 \leqslant r \leqslant d-1\Rightarrow r \in I$. Since $d$ is the smallest positive integer in $I$, then it must be the case that $r = 0$, by definition of the ideal. Therefore, $I = (d)$.  
\end{proof}

\section{Ring homomorphisms}

\begin{definition}
  Let $A, B$ two commutative rings. Then, $f: A \to B$ is a \emph{ring homomorphism} if, for $a, b \in A$ and well-defined operations
  \begin{gather*}
    f(a + b) = f(a) + f(b) \\
    f(ab) = f(a)f(b) \\
    f(1_A) = 1_B \\
  \end{gather*}
\end{definition}
\begin{theorem*}
  Let $f: A \to B$ be a ring homomorphism. Then, $\ker f$ is an ideal, and $\operatorname{im} f$ is a subring.
\end{theorem*}

\begin{definition}
  A \emph{subring} is a subset of a ring that is a ring with respect to the same operations and constants as the superset. 
\end{definition}

\aparte{Examples}{
  Consider $C$ an arbitrary subring of $\mathbb{Z}$. Then, it must contain $0, 1$, and be defined over the same $+, \cdot$. Therefore, since it must be closed under addition, it must contain $-1$, as well as
  \[
    \underbrace{1 + 1 + ... + 1}_{n}
  \]
  for any $n \in \mathbb{Z}$. This means that $C = \mathbb{Z}$.

  For a more interesting example, take a ring homomorphism $f: \mathbb{Z}/n\mathbb{Z} \to \mathbb{Z}/m\mathbb{Z}$, with $n, m, \in \mathbb{N}$. We can make a few observations about it.
  \begin{itemize}
    \item $f$ cannot map to a proper subring of $\mathbb{Z}/m\mathbb{Z}$. We know that $[1]_m \in \operatorname{im} f$, meaning that 
    \[
      \underbrace{[1]_m + ... + [1]_m}_{k \text{ times}} = [k]_m \in \operatorname{im} f \forall k < m
    \]
    Therefore, $\operatorname{im} f = \mathbb{Z}/m\mathbb{Z}$. 
    \item $f([n]_n) = f([0]_n) = [0]_m$, but also, $f([n]_n) = f([1]_n \cdot n) = [1]_m \cdot n = [n]_m$, meaning that $[n]_m = 0 \Rightarrow n \equiv 0 (\mod m)$, which implies $m|n$. 
    \item Additionally, $f$ is such that $[1]_n \stackrel{f}{\mapsto} [1]_m$, meaning that $[k]_n \stackrel{f}{\mapsto} [k]_m$ for any $k$. This implies that $f$ is unique. 
  \end{itemize}
  
  In conclusion: 
  \begin{theorem*}
    There exists a ring homomorphism $f: \mathbb{Z}/n\mathbb{Z} \to \mathbf{Z}/m\mathbb{Z}$ if and only if $m|n$. In that case, then $f$ is unique, and its image is equal to $\mathbb{Z}/m\mathbb{Z}$. 
  \end{theorem*}
}

Now for A Quick Tour of This Land's Characteristic Rings. 

Let $A$ a ring. There exists only a single ring homomorphism $\tau: \mathbb{Z} \to A$. This is due to the fact that $\tau(0) = 0, \tau(1) = 1_A \Rightarrow \tau(k) = \tau(1 + 1 + ... + 1) = 1_A + ... + 1_A = k \cdot 1_A \in A$. Therefore, the mapping is uniquely determined, and $\tau(n \cdot k) = \tau(n) \cdot \tau(k)$. 

There are two possibilities to characterize the kernel of this transformation, either $\ker \tau = (0)$, or $\ker \tau = (d)$ with $d \geqslant 2$: in short, it cannot be 1.\footnote{This makes sense in accordance to the definition of a ring homomorphism: $\tau(1) \neq 0$ for any $\tau$}. Let us formalize this: 

\begin{definition}
  Let $A$ be a ring, $\tau$ the unique homomorphism from the integers to it. Then, the \emph{characteristic} of $A$ is 
  \[
    c_A = \begin{cases}
      0 &, \ker \tau = 0 \\
      d &, \ker \tau = d \geqslant 2
    \end{cases}
  \]
\end{definition}
\aparte{Examples}{
  The characteristic of the real numbers is 0, since the field homomorphism: 
  \begin{align*}
    \tau : \mathbb{Z} &\to \mathbb{R} \\
                n     &\mapsto n
  \end{align*}
  maps 0 (and only 0) to 0. Therefore, its kernel is $\{0\}$, and $C_\mathbb{R} = 0$. A similar argument gives us that $C_\mathbb{Z} = 0$, and we can consider
  \begin{align*}
    \tau : \mathbb{Z} &\to \mathbb{Z}/n\mathbb{Z} \\
              k       &\mapsto [k]_n
  \end{align*}
  Which maps every multiple of $n$ to 0. Therefore, $\ker \tau = (n)$, and $C_{\mathbb{Z}/n\mathbb{Z}} = 0$. 
}

\begin{theorem*}
  $A$ is an integral domain $\Rightarrow c_A = 0$ or $c_A  = p$ a prime.  
\end{theorem*}
\begin{proof}
  Suppose $c_A = m \cdot k$ for $m, k > 1$. Then, $\tau(m) \cdot \tau(k) = \tau(m \cdot k) = 0 \in A$, meaning that there are two nontrivial zero divisors in $A$. 
\end{proof}
This holds for fields as well, since they are integral domains necessarily. 

\begin{definition}
  Let $A, B$ be two rings. Then, the \emph{direct product} is given by $A \times B = \{(a, b), a \in A, b \in B\}$. It has ring structure, with neutral elements $(0_A, 0_B), (1_A, 1_B)$, and component wise operations. 
\end{definition}

\aparte{Examples}{
  Consider the ring $A = \mathbb{Z}/n\mathbb{Z} \times \mathbb{Z}/m\mathbb{Z}$, and let us compute its characteristic. We first describe the homomorphism: 
  \begin{align*}
    \tau : \mathbb{Z} &\to \mathbb{Z}/n\mathbb{Z} \times \mathbb{Z}/m\mathbb{Z}
  \end{align*}
  such that $\tau(1) = ([1]_n, [1]_m)$. Therefore, we know that
  \[
    \tau(k) = ([k]_n, [k]_m) = ([0]_n, [0]_m) \Leftrightarrow k \equiv 0\ (\mod n) \land k \equiv 0\ (\mod m)
  \]
  The characteristic is the smallest element of such form: $c_A = \lcm(n, m)$. 

  This can be generalized to the characteristic of any product of two rings $A, B$: $c_{A \times B} = \lcm(c_A, c_B)$. This lets us construct fields of prime characteristics, but that aren't fields: 
  \[
    A = \mathbb{Z}/p\mathbb{Z} \times \mathbb{Z}/p\mathbb{Z}
  \]
  has prime characteristic $c_A = \lcm(p, p) = p$, but has non-trivial zero divisors as well: 
  \[
    (0, 1) \cdot (0, 1) = (0, 0)
  \]
  Which means it is not an integral domain. 
}

\aparte{Method}{
  Let's compute more characteristics! 
  
  Let $B = \mathbb{Z} \times \mathbb{Z}/n\mathbb{Z}$. Then, using $\tau(k) = (k, [k]_n)$, we can see that the kernel of $\tau$ must be equal to $\{0\}$, since no other term will leave a 0 in the first spot. Since $\ker \tau = (0)$, then $c_B = 0$. 

  Let $D = \mathbb{Z}/n\mathbb{Z}[x]$ be the ring of polynomials over $x$ with coefficients in $\mathbb{Z}/n\mathbb{Z}$. The homomorphism must be such that $\tau(1) = [1]_n$, meaning that $\tau(k) = [k]_n = 0 \Leftrightarrow n|k$. This means that $\ker \tau = (n)$, meaning that $c_D = n$. 
}

\marginpar{Week 10}
\chapter{Proofs from exercise sets}
This barely needs such pompous titles but oh well. It's fun. 

\section{Theorem \ref{thm1}}
\begin{theorem*}
  Strong Induction $\Rightarrow$ Well-ordering principle.
\end{theorem*}
\begin{proof}
  We can prove this by induction. Suppose there exists a subset $Y \subset \mathbb{N}$ such that it contains no least element. Consider $P(n) = \text{''}n \notin Y\text{''}$.
  \begin{itemize}
    \item[] \textbf{Base:} If 0 was in $Y$, then it would be its least element, since there are no smaller elements of $\mathbb{N}$. As such, it cannot be that $0 \in Y$, meaning $P(0)$ is true.
    \item[] \textbf{Induction:} Assume $P(k)$ is true for any $k \in \{0, 1, ..., n\}$. Then, if it was in $Y$, $n + 1$ would be its smallest element, since every smaller element is not in $Y$. As such, $P(n + 1)$ holds as well. 
    Since $P$ is hereditary and true for 0, it is true for any $n \in \mathbb{N}$.
  \end{itemize}
\end{proof}

\section{Theorem \ref{primefact}}
\begin{theorem*}
  Any $n > 1$ can be expressed by the product of primes.
\end{theorem*}
\begin{proof}
  Consider $S = \{n \in \mathbb{N}: n > 1 \land n \text{ does not have a prime factorization}\}$. Then, we take the smallest element in this set, $k$. $k$ cannot be a prime, so there exists $a, b < k$ such that $k = ab$. However, $a$ and $b$ cannot be in $S$, since they are smaller than $k$. This means that $a = \prod {p_{a,k}}^{a_k}, b = \prod {p_{b, k}}^{b_k}$, meaning that their product is the product of primes. Therefore, $k$ cannot be in $S$, so $S$ has to be empty.
\end{proof}

\section{Theorem \ref{disjointorb}}
\begin{theorem*}
  Let $E$ a finite set, $x, y \in E$, $G$ a finite group. Then, 
  \[
    \Orb_x = \Orb_y \text{ ou } \Orb_x \cap \Orb_y = \emptyset
  \]
\end{theorem*}
\begin{proof}
  Assume that there exists an $s \in Orb_x : s \in Orb_y$. Then, there exist $g_x, g_y \in G$ such that
  \[
    s = g_x \cdot x = g_y \cdot y \Leftrightarrow x = g_x^{-1}g_y \cdot y = g \cdot y
  \]
  meaning that $x \in \Orb_y$, which in turn means $\Orb_x = \Orb_y$. Therefore, if there is an intersection between two orbits, then they must be the same. 
\end{proof}
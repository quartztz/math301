\documentclass{report} % can be changed to article 

% imports

\usepackage{graphicx}
\usepackage{amsmath}
\usepackage{amsthm}
\usepackage{amsfonts}
\usepackage{amssymb}
\usepackage[normalem]{ulem}
\usepackage[margin=1.5in]{geometry} % margin can be changed
\usepackage{changepage}
\usepackage{mdframed}
\usepackage{xcolor}
\usepackage{hyperref}

% environments

\newmdenv[
  topline=false,
  bottomline=false,
  rightline=false,
  skipabove=\topsep,
  skipbelow=\topsep
]{siderules}

\newcommand{\aparte}[2]{
  \vspace{0.5em}\noindent
  \begin{adjustwidth}{24pt}{0pt}
    \begin{siderules}
      \leavevmode\reversemarginpar{\raggedleft \textit{#1}} #2
    \end{siderules}
  \end{adjustwidth}
  \vspace{0.75em}
}

\newcommand{\sep}{
  \begin{center}
    \noindent\rule{0.35\textwidth}{0.5pt}
  \end{center}
}

% configs

\def\labelitemi{\ensuremath{\triangleright}}
\renewcommand{\theenumi}{\small {(\roman{enumi})}}
\definecolor{blueish}{RGB}{80, 150, 150}
\definecolor{redish}{RGB}{160, 70, 60}
\renewcommand{\emph}[1]{{\color{blueish} #1}}

% specific algebra stuff

\AtBeginEnvironment{pmatrix}{\setlength{\arraycolsep}{3pt}}

\newcommand{\tot}{\ensuremath\varphi}
\newcommand{\gen}[1]{\ensuremath\langle #1 \rangle}
\newcommand{\Orb}{\ensuremath\operatorname{Orb}}
\newcommand{\Stab}{\ensuremath\operatorname{Stab}}
\newcommand{\sgn}{\ensuremath\operatorname{sgn}}
\newcommand{\ord}{\ensuremath\operatorname{ord}}
\newcommand{\lcm}{\ensuremath\operatorname{lcm}}
\renewcommand{\mod}{\ensuremath\operatorname{mod}}
\renewcommand{\emptyset}{\varnothing}
\renewcommand{\bar}{\overline}

% specific maths stuff
\newtheorem*{theorem*}{Proposition}
\newtheorem*{unnumthm}{Theorem}
\newtheorem{theorem}{Theorem}[chapter]
\newtheorem{corollary}[theorem]{Corollary}
\theoremstyle{plain}
\newtheorem{example}{Example}
\newtheorem{definition}[theorem]{Definition}

\title{Algebra - MATH310}
\author{Jacopo ``quartztz'' Moretti}
\date{January 2024}

\begin{document}
  \maketitle

  \chapter*{Preface}
  Helo! I'm Jack :3.

  I'm a student that needs to type out courses in order to make sure they properly understand them. So I put them out into the world! They might help you more than they help me :D. They are given as they are, with no guarantee of quality but guarantee of goodwill, bla bla bla. You know the gist of it. 

  \section*{Elements of notation}

  The group operation for a group $G$ will usually be denoted $\cdot_G$, and its neutral element will be $e_G$. The index will be removed when it can be inferred from context. 

  For now, I'll denote $\mathbf{n} = \{1, ..., n\}$, because it's clunky to type and it's my notes, goddamnit. I might change it back at the end.

  Also sometimes I get caught up and I use the weird (though admittedly concise) notation:
  \begin{center}
    ``... consider an $0 \neq a \in A$''
  \end{center}
  which is to be read as ``consider an $a$ in $A$ such that it is non zero''. Sorry for the confusion.

  \tableofcontents

  \pagebreak

  \chapter{Introduction}
  \marginpar{Week 1}
  Algebra rests on 3 basic principles, which are equivalent in nature. 
  \begin{enumerate}
    \item \textbf{Induction:} Let $S \subset \mathbb{N}$ such that $0 \in S$ and $n \in S \Rightarrow n + 1 \in S$. Then, $S = \mathbb{N}$. 
    \item \textbf{Well-ordering principle:} For any non-empty $A \subset \mathbb{N}$, there exists an element $a: \forall b \in A, a \leqslant b$. 
    \item \textbf{Strong induction:} Let $S \subset \mathbb{N}$ such that $0 \in S$ and $\{0, ..., n\} \in S \Rightarrow n + 1 \in S$. Then, $S = \mathbb{N}$. 
  \end{enumerate}

  It is well-established that these three principles are equivalent. Let us prove it. 

  \begin{theorem}\label{thm1}
    \textbf{I} $\Rightarrow$ \textbf{WOP} $\Rightarrow$ \textbf{SI} $\Rightarrow$ \textbf{I}.
  \end{theorem}
  \begin{proof}
    We will prove each induction separately.
    \begin{enumerate}
      \item $1. \Rightarrow 3.$ Let $S$ be the construction from the strong induction definition, and let us consider $P(n) = \{0, 1, ..., n\} \subset S$. We can prove it by induction: 
      \begin{itemize}
        \item[] \textbf{Base:} $0 \in S$ by construction $\Rightarrow \{0\} \subset S$. 
        \item[]\textbf{Induction:} Let us prove that $P(k) \Rightarrow P(k + 1)$ for some $k$.
        \begin{align*}
          \{0, 1, ..., k\} \subset S \text{ [by IH]} 
          &\Rightarrow k \in S &\text{ [by construction]} \\
          &\Rightarrow k + 1 \in S &\text{ [by definition]} \\
          &\Rightarrow \{0, 1, ..., k, k + 1\} \in S &
        \end{align*}
        Since it is hereditary and true for $0$, it is true $\forall n \in \mathbb{N}$ by the induction principle. 
      \end{itemize} 
      Since $\{0, 1, ..., n\} \subset \mathbb{N} \ \forall n$, then $S = \mathbb{N}$.
      \item $2 \Rightarrow 1$. Suppose $S \subset \mathbb{N}$ such that $0 \in S$ and $n \in S \Rightarrow n + 1 \in S$. Consider $S' = \mathbb{N} \setminus S$, which we assume to be nonempty by absurd. By the well-ordering principle, we can pick a least element in $k \in S'$, which is by definition not in $S$. $k$ cannot be zero, since $0 \in S$ by definition, but it can also not be non-zero, since $k \neq 0 \Rightarrow k = m + 1$ for some $m < k$ (therefore not in $S'$). $m \in S$, so by construction, $m + 1 = k \in S$ as well, which is a contradiction. $S'$ has to be empty, so $S = \mathbb{N}$. 
      \item $3 \Rightarrow 2$. Done in a Problem Set, found in appendix A.
    \end{enumerate}
  \end{proof}

  \chapter{Primes}

\section{Divisors and primes}

\begin{definition}
  Let $a, b \in \mathbb{Z}$. We say that $a$ \emph{divides} $b$ (notate: $a|b$) if there exists $k \in \mathbb{Z}$ such that $b = ka$.
\end{definition}
\begin{definition}
  A number $p \in \mathbb{Z}$ is prime if $p > 1$ and the only numbers that divide it are itself and 1.
\end{definition}

\begin{theorem}
  Any $n > 1$ has a prime divisor.
\end{theorem}
\begin{proof}
  Let $S = \{n \in \mathbb{N}: n > 1 \land n\text{ has no prime divisors}\}$. We suppose $S$ to be nonempty, meaning it contains a least element $k \in S$. $k$ cannot be prime, since $k|k \ \forall k$. Therefore, it has to be true that $k = ab$ for $a, b < k \in \mathbb{N}$. Since $k$ was the lest element, then, $a \notin S$, meaning that there exists a prime $p$ such that $a = pt$ for $t in \mathbb{N}$. Therefore, $k = ab = ptb \Rightarrow p|k$, contradicting our construction of $S$. Therefore, $S$ must be empty.
\end{proof}
\begin{theorem}\label{primefact}
  Any $n > 1$ can be expressed by the product of primes.
\end{theorem}
\aparte{}{This proof was done in an exercise set, and can be found in the appendix.}
\begin{theorem}
  The prime number factorization of a number is unique.
\end{theorem}
\begin{proof}
  Let $k = \prod^n p_i = \prod^m q_j$ two distinct prime sets. Suppose without loss of generality that $q_1 > p_1$ and let $t = (q_1 - p_1)q_2...q_m > 0$. Then: 
  \begin{align*}
    t 
    &= (q_1 - p_1)q_2...q_m \\
    &= q_1q_2...q_m - p_1q_2...q_m \\
    &= k - p_1q_2...q_m > 0 \Rightarrow p_1|t
  \end{align*}
  We know that $p_1 \neq q_j$ for all $j$, so we focus on the only ``weird'' term: 
  \begin{align*}
    (q_1 - p_1) &= sp_1 \\
    \Rightarrow q_1 &= (s + 1)p_1
  \end{align*}
  Which is a contradiction because $q_1$ is supposed to be prime. Therefore, the prime factorization is unique. 
\end{proof}

\section{Integer arithmetic}

\begin{definition}[Euclidian division]
  Let $n \in \mathbb{Z}, d \in \mathbb{Z}^*$. There exists a unique pair $q, r \in \mathbb{Z}$ such that $n = qd + r$ with $0 < r < d$. 
\end{definition}
\begin{proof}
  \textbf{Existence.}
  Consider the set of all numbers $S = \{n - kd\}_{k \in \mathbb{Z}} \cap \mathbb{N} = \{n - kd, kd \leqslant n\}_{k \in \mathbb{Z}}$. 

  We know that $S$ is not empty, because: 
  \begin{itemize}
    \item if $n >= 0$, then we set $k = 0$, meaning $n \in S$
    \item if $n < 0$, then we set $k = |n| + 1$, meaning $kd > |n|$ and $n + kd \in S$.
  \end{itemize}
  Since it's never empty, we can pick the least element of $S$ by means of the well-ordering principle. Let's call it $r$. Therefore, we have $r = n - kd$ for some $k$. To prove $r < d$, we assume towards absurdity that $r >= d$, meaning that 
  \[
    n - (k + 1)d = n - kd - d = r - d >= 0
  \]
  meaning $r$ wasn't minimal, which is a contradiction.

  \textbf{Uniqueness.} Suppose $n = q_1d + r_1 = q_2d + r_2$. Without loss of generality, assume $q_1 > q_2$. Then:
  \[
    (q_1 - q_2)d + r_1 = r_2 \geqslant d
  \]
  Since $r_1$ and $q_1 - q_2$ are positive. This contradicts the definition of $r_2$, and is therefore absurd.
\end{proof}

\begin{definition}
  Let $a, b \in \mathbb{Z}$. We define the greatest common divisor ($\gcd$) of two numbers as 
  \[
    \gcd(a, b) = \max\{x \in \mathbb{Z}: x|a \land x|b\}
  \]
\end{definition}
\begin{theorem}
  For $n, q \in \mathbb{Z}, d \in \mathbb{Z}^*$, such that $n = qd + r$, it is always the case that:  
  \[ \gcd(n, d) = \gcd(d, r) \]
\end{theorem}
\begin{proof}
  By inspection of the relationship $n = qd + r$, it's clear that if $x | n \land x | d$ then $x | r$, and if $x | d \land x | r$ then $x | n$.
\end{proof}

\aparte{Method}{
  This induces a special algorithm to compute the gcd of two numbers! Let $d_1, d_2 \in \mathbb{Z}$. Then: 
  \begin{align*}
    d_1 &= q_1 d_2 + d_3 \\
    d_2 &= q_2 d_3 + d_4 \\
    &... \\
    d_k &= q_kd_{k + 1} + 0
  \end{align*}
  The relationship $\gcd(d_{i - 1}, d_i) = \gcd(d_i, d_{i + 1})$ holds down the tree, meaning that by the end
  \[
    \gcd(d_1, d_2) = d_{k + 1}
  \]
  Additionally, we have: 
  \begin{corollary}\label{gcdab}
    For any $a, b \in \mathbb{Z}^+$, there exist $x, y \in \mathbb{Z}$ such that 
    \[
      \gcd(a, b) = xa + yb
    \]
  \end{corollary}
  This is obtained by running Euclid ``up the tree''. 
  \begin{example}
    TODO
  \end{example}
}

Special consequence of corollary \ref{gcdab} is the following
\begin{corollary}
  If $a, b \in \mathbb{Z}^+$ are such that $d = \gcd(a, b)$, then the equation: 
  \[
    c = ax + by
  \]
  has solutions $(x, y)$ if and only if $\exists \ k > 0: c = kd$, and they can be found as the solutions in corollary \ref{gcdab} multiplied by $k$.
\end{corollary}
Final consequence of these facts is the well-known Bézout's theorem.
\begin{theorem}
  Two numbers $a, b \in \mathbb{Z}^+$ are relatively prime if and only if the equation
  \[
    1 = ax + by
  \] has integer solutions.
\end{theorem}

\begin{definition}
  For any $n \in \mathbb{Z}^+$, Euler's totient function is defined as: 
  \[
    \varphi(n) = \big|\big\{k \in \{1, ..., n\}: \gcd(k, n) = 1\big\}\big|
  \]
  meaning the number of positive integers less than $n$ that are coprime to it.
\end{definition}
\aparte{Properties}{
  Properties of the totient function include: 
  \begin{itemize}
    \item $\varphi(p) = p - 1$ for any prime $p$.
    \item $\varphi(pq) = (p - 1)(q - 1)$ for any pair of distinct primes $p, q$. 
    \item More generally, $\varphi(mn) = \varphi(m)\varphi(n)$ for any $m, n$ coprime.
  \end{itemize}
}
  \chapter{Groups}
\marginpar{Week 2}

\section{Base definitions}

\begin{definition}
  A \emph{group} is a set $G$ with a binary operation $\cdot: G \times G \to G$, satisfying the following axioms: 
  \begin{itemize}
    \item $\cdot$ is \emph{associative}: $(a \cdot b) \cdot c = a \cdot (b \cdot c)$
    \item There exists a \emph{neutral element} $e$ such that $a \cdot e = e \cdot a = a \ \forall a \in G$. 
    \item For any $a \in G$ there exists an \emph{inverse} $a^{-1}$ such that $a^{-1} \cdot a = a \cdot a^{-1} = e$. 
  \end{itemize}
  We say that $G$ is a \emph{finite} group if $|G| < \infty$. In that case, we say that $G$ is of \emph{order} $|G|$. 
  We say that $G$ is \emph{abelian} (or commutative) if $a\cdot b = b\cdot a \ \forall a, b \in G$. 
\end{definition}
\begin{definition}
  $H \subset G$ is a \emph{subgroup} if it contains the neutral element $e_G$ and if it is closed with respect to $\cdot_G$, meaning that for every $a, b \in H$, $a\cdot b \in H$, and to inverses. 
\end{definition}

We can note that any group has a subgroup generated by a single element:
\[
  \langle g \rangle = \{e, g^{1}, g^{2}, \dots, g^{-1}, g^{-2}, \dots\}
\]
Since $g^i \cdot g^j = g^{i + j}$ by definition of the group operation, this set is closed under it, meaning it is a subgroup. 

\begin{definition}
  If it exists, the minimal $n \in \mathbb{N}^*$ such that $g^n = e$ is called the \emph{order} of $g$. It is finite for every element in a finite group. 
\end{definition}

\section{Cosets}

\begin{definition}
  Let $H\subset G$ be a subgroup of G. The \emph{left coset} of $g$ with respect to $H$, denoted $gH$, is the following set: 
  \[
    gH = \{gh, h \in H\}
  \]
\end{definition}

\begin{theorem}
  Let $H \subset G$ finite. Then: 
  \begin{enumerate}
    \item Two left-cosets $xH, yH$ are either disjoint ($xH \cap yH = \emptyset$) or equal. 
    \item For any element $g \in G$ there exists a left coset of $H$ such that $g \in H$.
    \item $|xH| = |H| \ \forall x \in G$
  \end{enumerate}
\end{theorem}
\begin{proof}
  We will prove each part separately: 
  \begin{enumerate}
    \item Suppose $xH, yH$ are such that $xH \cap yH \neq \emptyset$. This means that there exist $h_1, h_2$ such that $xh_1 = yh_2$. Therefore, 
    \[
      x = yh_2h_1^{-1} = yh_3 \in yH \Rightarrow xh = yh_3h \ \forall h \in H
    \]
    This means that if there exists an element of $xH$ that is in $yH$, then every element in $xH$ can be written as an element in $yH$, meaning they are equal.
    \item For any $g \in G$, one can construct $gH = \{e, g, g^2, ...\}$, which naturally contains $g$. 
    \item The mapping 
    \begin{align*}
      f(h): H     &\to xH \\
            h &\mapsto xh
    \end{align*}
    is surjective, by definition of $xH = \{xh, h \in H\}$, and it is also injective, since $xh_1 = yh_2 \Leftrightarrow h_1 = h_2$. This means it defines a bijection between $H$ and $xH$, indicating they have the same cardinality.
  \end{enumerate}
  \aparte{Example}{
    Let $G = (\mathbb{Z}, +, 0), H = 3\mathbb{Z}\subset \mathbb{Z}$. The left coset of 0 with respect to $H$ is : 
    \[
      \{0 + 3k\}_{k \in \mathbb{Z}} = H = \{3 + 3k\}_{k \in \mathbb{Z}}
    \]
    The left coset of 1 is 
    \[
      \{1 + 3k\}_{k \in \mathbb{Z}} = \{1, 4, 7, -2, ...\}
    \]
  }
\end{proof}

\begin{theorem}[Lagrange]
  Let $G$ be a finite group, $H \subset G$ a subgroup. Then, $|H|$ divides $|G|$.
\end{theorem}
\begin{proof}
  Each $g \in G$ belongs to a left coset of $H$, which are either disjoint or equal. This means:
  \begin{align*}
    G &= \bigcup_{i = 0}^{r}x_iH &[\text{disjoint union of finitely many sets}] \\
    \Rightarrow |G| &= \sum_{i = 0}^r |x_iH| &\\
    \Rightarrow |G| &= \sum_{i = 0}^r |H| &[\text{since } |xH| = |H|] \\
    \Rightarrow |G| &= r|H| &
  \end{align*}
  with $r \in \mathbb{N}$, meaning that $|H|$ divides $|G|$. 
\end{proof}

\begin{definition}
  The number of left cosets of $H$ of $G$ is called the \emph{index} of $G$: 
  \[
    [G : H] = |G|/|H| \in \mathbb{N}^*
  \]
\end{definition}

This means that the order of any element $g \in G$ (notated $\operatorname{ord}(g)$) divides the order of the group $|G|$, since every element generates a subgroup $\langle g \rangle$. Additionally, it implies 

\begin{corollary}\label{stupid}
  $g^{|G|} = (g^{\operatorname{ord}(g)})^{k} = e^k = e$ for some $k$.
\end{corollary}

\section{RSA}

\begin{theorem}[Euler's theorem]
  Let $a, n \in \mathbb{Z}^+$. such that $\gcd(a, n) = 1$. Then, \[ a^{\varphi(n)} \equiv 1 \ \operatorname{mod} n\]
\end{theorem}
\begin{proof}
  Consider $G = (\mathbb{Z}/n\mathbb{Z}, \cdot, 1)$. Then, 
  \[
    a^{\varphi(n)} = a^{|G|} \stackrel{\ref{stupid}}{=} 1
  \]
\end{proof}

\begin{theorem}[Fermat's little theorem]
  Let $a \in \mathbb{Z}^+$, $p$ prime such that $p$ does not divide $a$. Then, $a^{p - 1} = 1$. 
\end{theorem}
\begin{proof}
  Consider $G = (\mathbb{Z}/p\mathbb{Z}, \cdot, 1)$. Then, $|G| = \varphi(p) = p - 1$. By Euler's theorem, \[a^{\varphi(p)} = a^{(p - 1)} = 1\]
\end{proof}

\aparte{RSA}{
  The RSA cryptosystem for message transmission works as follows: 
  \begin{enumerate}
    \item Choose two distinct large primes $p, q$. 
    \item Compute $m = pq \Rightarrow \varphi(m) = (p - 1)(q - 1)$. 
    \item Choose $e \leqslant m$ an encryption key such that $\gcd(e, \varphi(m)) = 1$. 
    \item Use Euclid's algorithm to determine $d$ such that $ed - k\varphi(m) = 1$ for some integer $k$.
    \item The encoding key is the pair $(m, e)$, and it can be published. To decode, you use the decoding key $(m, d)$ which is to be kept private. 
  \end{enumerate}

  To send a message $x$ to someone, you need their public pair $(m, e)$. You first compute $c \equiv x^e \operatorname{mod} m$, which can be sent publicly. To decode, the person will use their private pair $(m, d)$, computing $x \equiv c^d \operatorname{mod} m \equiv x^{ed} \operatorname{mod} m$.
}

Why is it the case that $x^{ed} \equiv x \operatorname{mod} m$? Well...

\begin{theorem}
  Let $p, q$ be two distinct primes, and $m = pq$. Let $e: \gcd(e, \varphi(m)) = 1$, and let $d \in \mathbb{Z}: ed - k\varphi(m) = 1$ for some $k \in \mathbb{Z}$. Then, 
  \[
    x^{ed} \equiv x \operatorname{mod} m
  \]
  for all $x \in \{1, ..., m\}$.
\end{theorem}

  \chapter{Rings and fields}

\marginpar{Week 8}
\section{F*!king finally!}

\begin{definition}
  A \emph{ring} is a set $A$ with two operations defined on it: $+$ and $\cdot$, in such a way that
  \begin{enumerate}
    \item $A$ is an abelian group with respect to $+$, containing a neutral element $0$. 
    \item $\cdot$ is associative: $(a \cdot b) \cdot c = a \cdot (b \cdot c)$, et il $\exists 1 \in A: 1 \cdot a = a \cdot 1$.
    \item  $\cdot$ is both left- and right-distributive over $+$: $a \cdot (b + c) = ab + ac,  (a + b) \cdot c = ac + bc$. 
  \end{enumerate}
\end{definition}
\aparte{Examples}{
  Let us consider some examples: 
  \begin{itemize}
    \item $\mathbb{Z}: \gen{+, \cdot, 0, 1}$ is a ring! Though it doesn't contain any multiplicative inverses, it fits the definitions nicely. 
    \item $A = \mathbb{Z}[\sqrt{2}] := \{a + b\sqrt{2}, a, b \in \mathbb{Z}\}$ contains 0 and 1. Additionally: 
    \begin{gather*}
      (a + b\sqrt{2}) + (c + d \sqrt{2}) = (a + c) + (b + d)\sqrt{2} \in A\\
      -a - b\sqrt{2} \in A \\
      (a + b\sqrt{2})(c + d\sqrt{2}) = ac + 2bd + \sqrt{2}(ad + bc) \in A \\
    \end{gather*}
    \item $\mathbb{Q}[\sqrt{2}]$ is a ring. Proof of this is left as an exercice to the reader. 
  \end{itemize}
}

In this course, we will consider only \emph{commutative rings}, such that $ab = ba \forall a, b \in A$. 

\begin{definition}\label{zerodiv-def}
  $a \in A$ is a \emph{zero divisor} if there exists an $0 \neq x \in A$ such that $ax = 0$. 
\end{definition}
Most rings we are used to do not have any nontrivial zero divisors. Let us consider another one: 
\aparte{Example}{
  Consider: 
  \[
    \mathbb{Z}/n\mathbb{Z} = \{[0], [1], ..., [n - 1]\}
  \]
  It is isomorphic to the cyclic group of order $n$, meaning it is an abelian group with respect to addition. $\cdot$ is associative, and $[1]$ is its neutral element wrt multiplication. 

  Consider now an element $[a] \in \mathbb{Z}/n\mathbb{Z}$. 
  \begin{itemize}
    \item If $\gcd(a, n) = d > 1$, then we can write \[[a] \cdot \left[\frac{n}{d}\right] = [1]\] meaning that $a$ is a nontrivial zero divisor of the ring. 
    \item If $a, n$ are coprime, then by Bezout's theorem 
    \begin{gather*}
      \exists x, y : ax + ny = 1 \Rightarrow [a][x] = [1] \Rightarrow [a]^{-1} = [x]
    \end{gather*}
    Additionally, if we consider $[b]: [b] \cdot [a] = 0$, then we have
    \[
      [b] \cdot [a] = [0] \Rightarrow [b] \cdot [a] \cdot [x] = [0] \Rightarrow [b] \cdot [1] = [0] \Rightarrow [b] = [0]
    \]
    Meaning that $[a]$ cannot be a zero divisor of the ring. 
  \end{itemize}
  This means that an element of $\mathbb{Z}/n\mathbb{Z}$ is either invertible, or a zero divisor. 
}

\begin{definition}
  A ring that has no nontrivial zero divisors is called an \emph{integral domain}.
\end{definition}
\begin{definition}
  A commutative ring where all nonzero elements have a multiplicative inverse is called a \emph{field}. $\forall 0 \neq a \in A \exists a^{-1} \in A : a a^{-1} = a^{-1} a = 1$
\end{definition}
\begin{corollary}
  $\mathbb{Z}/n\mathbb{Z}$ either has nontrivial zero divisors ($\Leftrightarrow n$ is not prime) or it is a field and an integral domain ($\Leftrightarrow n = p$ for some prime $p$). 
\end{corollary}
\begin{proof}
  If it has no nontrivial zero divisors, that means that there is no $[a] \in \{[1], ..., [n - 1]\}$ such that $\gcd(a, n) > 1$. Therefore, $n$ has no divisors other than itself and 1, and it must be a prime $p$ by definition. If that's the case, then $\forall [b] \neq 0, \gcd(b, p) = 1 \Rightarrow [b]$ has a multiplicative inverse, meaning that $\mathbb{Z}/n\mathbb{Z}$ is a field.
\end{proof}

\begin{theorem*}
  A field is an integral domain. An invertible element is not a zero divisor. 
\end{theorem*}
\begin{proof}
  Suppose $ab = 0$ in a field $A$ such that $a \neq 0$. This means that there exists a $a^{-1} \in A$ such that $aa^{-1} = 1$. This means that :
  \begin{align}
    ab = 0 
    &\Leftrightarrow a^{-1}ab = a^{-1} \cdot 0 \\
    &\Leftrightarrow 1 \cdot b = 0 \\
    &\Leftrightarrow b = 0
  \end{align}
  meaning $a$ cannot be a zero divisor. This means that if all non-zero elements in $A$ are invertible, then $A$ cannot have nontrivial zero divisors. This is equivalent to saying that if $A$ is a field, then it is an integral domain, proving our proposition.
\end{proof}
\aparte{Remark}{The converse isn't true! $\mathbb{Z}$ is an integral domain, but $2 \in \mathbb{Z}$ does not have a multiplicative inverse.}

This allows the following characterization: 
\[
  \text{Fields} \subset \text{Integral Domains} \subset \text{Commutative Rings}
\]
And for the following statements on $\mathbb{Z}/n\mathbb{Z}$ : 
\[
  \mathbb{Z}/n\mathbb{Z} \text{ integral domain} \Leftrightarrow \mathbb{Z}/n\mathbb{Z} \text{ field} \Leftrightarrow n = p \text{ prime}
\]

\section{Less than ideal}

\begin{definition}
  Let $A$ be a commutative ring. $I \subset A$ is an \emph{ideal} if it has the following properties:
  \begin{itemize}
    \item it is a subgroup with respect to +, i.e: it contains 0, $-a$ for any $a \in I$, and it is closed wrt +. 
    \item $\forall x \in A, a \in I: x \cdot a \in I$. 
  \end{itemize}
\end{definition}
\aparte{Examples}{
  Let's have a look at some examples
  \begin{itemize}
    \item $\{0\} \subset A, A \subset A$ are ideals for any commutative ring $A$. That is not very interesting.
    \item Consider $\mathbb{Z}$. Then, $2\mathbb{Z} = \{2a, a \in \mathbb{Z}\}$ is an ideal! It's easy to verify it is a subgroup for $+$, and we have that, for any $x \in \mathbb{Z}, 2a \cdot x \in 2\mathbb{Z}$. This is true for any $d \in \mathbb{Z}$. 
  \end{itemize}
}

\begin{definition}
  We say that an ideal $I \subset A$ is \emph{proper} if $I \neq A$, and \emph {nontrivial} if $I \neq 0$. 
\end{definition}
\aparte{Properties}{
  Let $I, J \subset A$ two ideals. Then: 
  \begin{enumerate}
    \item[(i)] if $1 \in I$, then $I = A$. 
    
    \textit{This is due to the properties of ideals: $\forall x \in A, 1 \cdot x \in I \Rightarrow x \in I \Rightarrow I = A$}
    
    \item[(ii)] $I \cap J$ is an ideal

    \textit{Again, easy to verify applying properties of ideals: let $x, y \in I \cap J$.}
    \begin{gather*}
      x + y \in I, J \Rightarrow x + y \in I \cap J \\
      -x \in I, J \Rightarrow -x \in I \cap J \\
      0 \in I, J \Rightarrow 0 \in I \cap J
    \end{gather*}
    \textit{Meaning it is a subgroup. Also, for any $a \in A$, we have that $ax \in I, ax \in J$, which means $ax \in I \cap J$. This makes it an ideal!}

    \item[(iii)] $I + J = \{i + j\}_{i \in I, j \in J}$ is an ideal.

    \item[(iv)] $I \cdot J = \{\sum_{i = 1}^k x_i y_i\}_{x_i \in I, y_i \in J}$ is an ideal.\footnotemark

    \item[(v)] $I \cup J$ is not necessarily an ideal

    \textit{More often than not, it is not an additive subgroup: consider $I = 2\mathbb{Z}$, $J = 3\mathbb{Z}$}
  \end{enumerate}
}
\footnotetext{Proofs to (iii) and (iv) are left as exercice to the reader, as they are similar to that of (ii).}

We can verify some fun properties on integer ideals. Let $A = \mathbb{Z}, I = 6\mathbb{Z}, J = 10\mathbb{Z}$. Then: 
\begin{itemize}
  \item \textbf{Intersection:} We have: 
  \begin{align*}
    I \cap J 
    &= \{z \in \mathbb{Z}: z = 6n \land z = 10m, n, m \in \mathbb{Z}\} \\
    &= \{\text{common multiples of 6 and 10}\} \\
    &= \{\text{all multiples of} \lcm(6, 10) = 30\} \\
    &= 30\mathbb{Z}
  \end{align*}
  \item \textbf{Addition:} We have: 
  \begin{align*}
    I + J 
    &= \{6n + 10m\}_{n, m \in \mathbb{Z}} \\
    &= \{\gcd(6, 10)k, k \in \mathbb{Z}\} \tag*{\text{[By Bezout]}} \\
    &= 2\mathbb{Z}
  \end{align*}
  \item \textbf{Product:} We have: 
  \begin{align*}
    I \cdot J 
    &= \left\{\sum_{i = 1}^k 6x_i \cdot 10y_i, x_i, y_i \in \mathbb{Z}\right\} \\
    &= \left\{ 60\sum_{i = 1}^k x_iy_i,  x_i, y_i \in \mathbb{Z} \right\} \\
    &= 60\mathbb{Z}
  \end{align*}
\end{itemize}

From which we can conclude!

\ 

\begin{unnumthm}
  Let $I = n\mathbb{Z}, J = m\mathbb{Z}$ two ideals of $\mathbb{Z}$. Then: 
  \begin{gather*}
    I \cap J = \lcm(n, m)\mathbb{Z} \\
    I + J = \gcd(n, m)\mathbb{Z} \\
    I \cdot J = (n \cdot m)\mathbb{Z}
  \end{gather*}
\end{unnumthm}

\aparte{Example}{
  Consider $A = \mathbb{R}[x]$ all polynomials on one variable with real coefficients\footnotemark. This forms a ring as is trivial to verify. Then, we consider the following: 
  \begin{gather*}
    I = \{(x + 5)f(x), f(x) \in \mathbb{R}[x]\} \\
    J = \{(x^2 + 2)f(x), f(x) \in \mathbb{R}[x]\}
  \end{gather*}
  which are comprised of all polynomials divisible by $(x + 5)$ and $(x^2 + 2)$, respectively. Then, we have:
  \begin{gather*}
    I \cap J = \{(x + 5)(x^2 + 2)f(x), f(x) \in \mathbb{R}[x]\} \\
    I \cdot J = \left\{\sum_{i = 0}^k (x + 5)f_i(x) \cdot (x^2 + 2)g_i(x)\right\} = \{(x + 5)(x^2 + 2)f(x)\} \\
    I + J = \{(x + 5) f(x) + (x^2 + 2) g(x)\}
  \end{gather*}
  We can find $f(x), g(x)$ such that $(x + 5) f(x) + (x^2 + 2) g(x) = 1 \in \mathbb{R}[x]$ (constant polynomial of value 1). For example, take 
  \[
    (x + 5)(x - 5)\left(\frac{-1}{27}\right) - (x^2 + 2)\left(\frac{-1}{27}\right) = (x^2 - 25 - x^2 - 2)\left(\frac{-1}{27}\right) = 1
  \]

  Can we do that for any pair of polynomials in $\mathbb{R}[x]$?
}
\footnotetext{If you want to see me get pedantic about notation for life half a page, go look at appendix C.}

\marginpar{Week 9}

\begin{definition}\label{princid}
  Let $S \subset A$. Let $I$ be the minimal ideal containing $S$. Then, we denote $I = (S)$ the \emph{ideal generated by the set} $S$. It means: 
  \[
    (S) = \left\{\sum_i s_i a\right\}_{s_i \in S, a \in A}
  \] 
  $I$ is called \emph{principal} if $I = (x) = \{xa, x \in I\}_{a \in A}$ is generated by a single element. 
\end{definition}
\aparte{Examples}{
  $A$ and $\{0\}$ are principal, being generated by $1$ and $0$ respectively. Additionally, $n\mathbb{Z} = (n)$ is principal. 
}

\begin{theorem*}
  $A$ is a field $\Leftrightarrow$ $A, \{0\}$ are the only ideals in $A$
\end{theorem*}
\begin{proof}
  We prove both directions separately: 
  \begin{itemize}
    \item[$\Rightarrow$] $A$ is a field, consider an ideal $I$ that isn't $\{0\}$ and an element $a \in I$, which is not 0. Since we're in a field, and it is non-zero, there exists $a^{-1} \in A$. By definition of an ideal, we have that $a a^{-1} \in I \Rightarrow 1 \in I \Rightarrow I = A$. 
    \item[$\Leftarrow$] Let $A$ be a ring such that $A$, $\{0\}$ are its only ideals. Therefore, we consider an $0 \neq a \in A$, and the ideal it generates. Since $a \neq 0$,we know that it must generate $A$, and that there must exist a $y \in A$ such that $ya = 1$, meaning that $y = a^{-1}$ by definition, and $A$ is a field as a consequence of its existence.  
  \end{itemize} 
\end{proof}

\section{Quotient rings}

\begin{definition}
  An \emph{equivalence relation} $\sim$ on a set $E$ is a relation such that for any $a, b, c \in E$, we have: 
  \begin{itemize}
    \item $a \sim a$, meaning it is \emph{reflexive}
    \item $a \sim b \Leftrightarrow b \sim a$, meaning it is \emph{symmetric}
    \item $a \sim b, b \sim c \Rightarrow a \sim c$, meaning it is \emph{transitive}
  \end{itemize}

  A \emph{congruence relation} $\sim$ on a commutative ring $A$ is an equivalence relation such that for any $a, b, c, d \in A$
  \[
    \begin{cases}
      a \sim b \\
      c \sim d
    \end{cases} \Rightarrow 
    \begin{cases}
      a + c \sim b + d \\
      ac \sim bd
    \end{cases}
  \]
\end{definition}

\begin{theorem*}
  If $I \subset A$ is an ideal, then the relation : 
  \[
    a \sim b \Leftrightarrow (b - a) \in I
  \]
  is an equivalence relation, and if it is a congruence relation, then the set $I = \{a \in A: a \sim 0\}$ is an ideal in $A$. 
\end{theorem*}
\begin{proof}
  We first check all of the properties of an equivalence relation. We have: 
  \begin{itemize}
    \item Any ideal contains the 0 element, meaning that $(a - a) \in I \Rightarrow a \sim a$. This makes $\sim$ reflexive. 
    \item $I$ must be an additive subgroup of $A$. Therefore, it must contain $b - a$ and $-(b - a) = a - b$, meaning that if $a \sim b$, then $b \sim a$. $\sim$ is therefore symmetric. 
    \item Consider $a, b, c \in A$ such that $a \sim b, b \sim c$. Therefore, this means that $c - b \in I, b - a \in I$. Now, consider that $c - a = (c - b) + (b - a)$ is the sum of two elements of $I$: since $I$ is an additive subgroup, that must mean that $c - a \in I$, meaning $a \sim c$. Therefore $\sim$ is transitive. 
  \end{itemize}
  Now that we have checked the properties of an equivalence relation, we must check those for a congruence relationship: consider $a \sim b, c \sim d$ in $A$. Then: 
  \[
    \begin{cases}
      (b + d) - (a + c) = (b - a) + (d - c) \in I \\
      bd - ac = (b - a)(d - c) = e(b - a) \in I
    \end{cases}
  \]
  Where the last step is motivated by the fact that $e = d - c$ is an element of $A$, and by the definition of an ideal. This concludes the proof of the first part of the theorem. 
  
  Then, consider $a \sim 0, b \sim 0$. We have: $a + b \sim 0, 0 \sim 0, -a \sim 0$, which completes the subgroup requirement, as well as $x \sim x \Rightarrow a \cdot x \sim 0 \cdot x \Rightarrow a \cdot x \sim 0 \Rightarrow \{a \in A: a \sim 0\}$ is an ideal! 
\end{proof}

\begin{theorem*}
  Let $A$ be a commutative ring, $\sim$ a congruence relation over $A$ such that $1 \not\sim 0$. Then, the set of congruence classes $A/\{x \in A : x \sim 0\} := A/{\sim}$ is a commutative ring. 
\end{theorem*}
\begin{proof}
  Notate $\bar{a} = \{x \in A: x \sim a\}$. Define $\bar{a} + \bar{b} = \bar{a + b}$, $\bar{a} \cdot \bar{b} = \bar{ab}$. These are well-defined thanks to the properties on $\sim$. Most importantly, $\bar{1} \in A/{\sim}$. 
\end{proof}

\aparte{Example}{
  Consider the ring of polynomials with real coefficients $\mathbb{R}[x]$. Let $I = \langle (x^2 - 4) \rangle$, and consider the commutative ring $B = \mathbb{R}[x]/I$. Within it, we have: 
  \[
    \bar{(x + 2)} \cdot \bar{(x + 1)} = \bar{x^2 + 3x + 2} = {\color{redish} \bar{x^2 + 3x + 2} - \bar{x^2 - 4}} = \bar{3x + 6}
  \]
  where the step in red is justified as a way to find the remainder of the function by the defined congruence relation. Another example: 
  \[
    \bar{x} \cdot \bar{x} = \bar{x^2} \mathbin{\color{redish}=} \bar{4}
  \]
  through a similar process as earlier. One can also look for the zero divisors (recall definition \ref{zerodiv-def}) within this ring. Consider: 
  \[
    \bar{(x + 2)}\bar{(x - 2)} = \bar{(x^2 - 4)} \mathbin{\color{redish}=} \bar{0}
  \]
  since $(x + 2), (x - 2)$ are non-zero, but their product is, this means that they are zero divisors, and that $B$ is not an integral domain. 
  
  As an exercice, one can show that any element in $B$ can be written under the form $\bar{ax + b}, a, b \in \mathbb{R}$. 
}

\begin{definition}
  Recall the definition of a principal ideal (Definition \ref{princid}). A commutative ring where every ideal is principal is called a \emph{principal ring}. An integral domain where every ideal is principal is called a \emph{principal integral domain} (PID). 
\end{definition}

This means that any field is a PID, since it only has two ideals, and they are both principal ($A$ is generated by 1, ${0}$ is generated by 0).

\begin{theorem*}
  $\mathbb{Z}$ is a PID. 
\end{theorem*}
\begin{proof}
  If $I = \{0\}$, then $I = (0)$. Suppose then that $I \neq \{0\}$. Therefore, there exists an $0 \neq a \in I$, such that $a \in I, -a \in I \Rightarrow |a| \in I$. Consider $d$ the smallest positive element in $I$. Then, for any $n \in I$, by euclidean division, $n = kd + r, 0 \leqslant r \leqslant d-1\Rightarrow r \in I$. Since $d$ is the smallest positive integer in $I$, then it must be the case that $r = 0$, by definition of the ideal. Therefore, $I = (d)$.  
\end{proof}

\section{Ring homomorphisms}

\begin{definition}
  Let $A, B$ two commutative rings. Then, $f: A \to B$ is a \emph{ring homomorphism} if, for $a, b \in A$ and well-defined operations
  \begin{gather*}
    f(a + b) = f(a) + f(b) \\
    f(ab) = f(a)f(b) \\
    f(1_A) = 1_B \\
  \end{gather*}
\end{definition}
\begin{theorem*}
  Let $f: A \to B$ be a ring homomorphism. Then, $\ker f$ is an ideal, and $\operatorname{im} f$ is a subring.
\end{theorem*}

\begin{definition}
  A \emph{subring} is a subset of a ring that is a ring with respect to the same operations and constants as the superset. 
\end{definition}

\aparte{Examples}{
  Consider $C$ an arbitrary subring of $\mathbb{Z}$. Then, it must contain $0, 1$, and be defined over the same $+, \cdot$. Therefore, since it must be closed under addition, it must contain $-1$, as well as
  \[
    \underbrace{1 + 1 + ... + 1}_{n}
  \]
  for any $n \in \mathbb{Z}$. This means that $C = \mathbb{Z}$.

  For a more interesting example, take a ring homomorphism $f: \mathbb{Z}/n\mathbb{Z} \to \mathbb{Z}/m\mathbb{Z}$, with $n, m, \in \mathbb{N}$. We can make a few observations about it.
  \begin{itemize}
    \item $f$ cannot map to a proper subring of $\mathbb{Z}/m\mathbb{Z}$. We know that $[1]_m \in \operatorname{im} f$, meaning that 
    \[
      \underbrace{[1]_m + ... + [1]_m}_{k \text{ times}} = [k]_m \in \operatorname{im} f \forall k < m
    \]
    Therefore, $\operatorname{im} f = \mathbb{Z}/m\mathbb{Z}$. 
    \item $f([n]_n) = f([0]_n) = [0]_m$, but also, $f([n]_n) = f([1]_n \cdot n) = [1]_m \cdot n = [n]_m$, meaning that $[n]_m = 0 \Rightarrow n \equiv 0 (\mod m)$, which implies $m|n$. 
    \item Additionally, $f$ is such that $[1]_n \stackrel{f}{\mapsto} [1]_m$, meaning that $[k]_n \stackrel{f}{\mapsto} [k]_m$ for any $k$. This implies that $f$ is unique. 
  \end{itemize}
  
  In conclusion: 
  \begin{theorem*}
    There exists a ring homomorphism $f: \mathbb{Z}/n\mathbb{Z} \to \mathbf{Z}/m\mathbb{Z}$ if and only if $m|n$. In that case, then $f$ is unique, and its image is equal to $\mathbb{Z}/m\mathbb{Z}$. 
  \end{theorem*}
}

Now for A Quick Tour of This Land's Characteristic Rings. 

Let $A$ a ring. There exists only a single ring homomorphism $\tau: \mathbb{Z} \to A$. This is due to the fact that $\tau(0) = 0, \tau(1) = 1_A \Rightarrow \tau(k) = \tau(1 + 1 + ... + 1) = 1_A + ... + 1_A = k \cdot 1_A \in A$. Therefore, the mapping is uniquely determined, and $\tau(n \cdot k) = \tau(n) \cdot \tau(k)$. 

There are two possibilities to characterize the kernel of this transformation, either $\ker \tau = (0)$, or $\ker \tau = (d)$ with $d \geqslant 2$: in short, it cannot be 1.\footnote{This makes sense in accordance to the definition of a ring homomorphism: $\tau(1) \neq 0$ for any $\tau$}. Let us formalize this: 

\begin{definition}
  Let $A$ be a ring, $\tau$ the unique homomorphism from the integers to it. Then, the \emph{characteristic} of $A$ is 
  \[
    c_A = \begin{cases}
      0 &, \ker \tau = 0 \\
      d &, \ker \tau = d \geqslant 2
    \end{cases}
  \]
\end{definition}
\aparte{Examples}{
  The characteristic of the real numbers is 0, since the field homomorphism: 
  \begin{align*}
    \tau : \mathbb{Z} &\to \mathbb{R} \\
                n     &\mapsto n
  \end{align*}
  maps 0 (and only 0) to 0. Therefore, its kernel is $\{0\}$, and $C_\mathbb{R} = 0$. A similar argument gives us that $C_\mathbb{Z} = 0$, and we can consider
  \begin{align*}
    \tau : \mathbb{Z} &\to \mathbb{Z}/n\mathbb{Z} \\
              k       &\mapsto [k]_n
  \end{align*}
  Which maps every multiple of $n$ to 0. Therefore, $\ker \tau = (n)$, and $C_{\mathbb{Z}/n\mathbb{Z}} = 0$. 
}

\begin{theorem*}
  $A$ is an integral domain $\Rightarrow c_A = 0$ or $c_A  = p$ a prime.  
\end{theorem*}
\begin{proof}
  Suppose $c_A = m \cdot k$ for $m, k > 1$. Then, $\tau(m) \cdot \tau(k) = \tau(m \cdot k) = 0 \in A$, meaning that there are two nontrivial zero divisors in $A$. 
\end{proof}
This holds for fields as well, since they are integral domains necessarily. 

\begin{definition}
  Let $A, B$ be two rings. Then, the \emph{direct product} is given by $A \times B = \{(a, b), a \in A, b \in B\}$. It has ring structure, with neutral elements $(0_A, 0_B), (1_A, 1_B)$, and component wise operations. 
\end{definition}

\aparte{Examples}{
  Consider the ring $A = \mathbb{Z}/n\mathbb{Z} \times \mathbb{Z}/m\mathbb{Z}$, and let us compute its characteristic. We first describe the homomorphism: 
  \begin{align*}
    \tau : \mathbb{Z} &\to \mathbb{Z}/n\mathbb{Z} \times \mathbb{Z}/m\mathbb{Z}
  \end{align*}
  such that $\tau(1) = ([1]_n, [1]_m)$. Therefore, we know that
  \[
    \tau(k) = ([k]_n, [k]_m) = ([0]_n, [0]_m) \Leftrightarrow k \equiv 0\ (\mod n) \land k \equiv 0\ (\mod m)
  \]
  The characteristic is the smallest element of such form: $c_A = \lcm(n, m)$. 

  This can be generalized to the characteristic of any product of two rings $A, B$: $c_{A \times B} = \lcm(c_A, c_B)$. This lets us construct fields of prime characteristics, but that aren't fields: 
  \[
    A = \mathbb{Z}/p\mathbb{Z} \times \mathbb{Z}/p\mathbb{Z}
  \]
  has prime characteristic $c_A = \lcm(p, p) = p$, but has non-trivial zero divisors as well: 
  \[
    (0, 1) \cdot (0, 1) = (0, 0)
  \]
  Which means it is not an integral domain. 
}

\aparte{Method}{
  Let's compute more characteristics! 
  
  Let $B = \mathbb{Z} \times \mathbb{Z}/n\mathbb{Z}$. Then, using $\tau(k) = (k, [k]_n)$, we can see that the kernel of $\tau$ must be equal to $\{0\}$, since no other term will leave a 0 in the first spot. Since $\ker \tau = (0)$, then $c_B = 0$. 

  Let $D = \mathbb{Z}/n\mathbb{Z}[x]$ be the ring of polynomials over $x$ with coefficients in $\mathbb{Z}/n\mathbb{Z}$. The homomorphism must be such that $\tau(1) = [1]_n$, meaning that $\tau(k) = [k]_n = 0 \Leftrightarrow n|k$. This means that $\ker \tau = (n)$, meaning that $c_D = n$. 
}

\marginpar{Week 10}

  \appendix
  
  \chapter{Proofs from exercise sets}
This barely needs such pompous titles but oh well. It's fun. 

\section{Theorem \ref{thm1}}
\begin{theorem*}
  Strong Induction $\Rightarrow$ Well-ordering principle.
\end{theorem*}
\begin{proof}
  We can prove this by induction. Suppose there exists a subset $Y \subset \mathbb{N}$ such that it contains no least element. Consider $P(n) = \text{''}n \notin Y\text{''}$.
  \begin{itemize}
    \item[] \textbf{Base:} If 0 was in $Y$, then it would be its least element, since there are no smaller elements of $\mathbb{N}$. As such, it cannot be that $0 \in Y$, meaning $P(0)$ is true.
    \item[] \textbf{Induction:} Assume $P(k)$ is true for any $k \in \{0, 1, ..., n\}$. Then, if it was in $Y$, $n + 1$ would be its smallest element, since every smaller element is not in $Y$. As such, $P(n + 1)$ holds as well. 
    Since $P$ is hereditary and true for 0, it is true for any $n \in \mathbb{N}$.
  \end{itemize}
\end{proof}

\section{Theorem \ref{primefact}}
\begin{theorem*}
  Any $n > 1$ can be expressed by the product of primes.
\end{theorem*}
\begin{proof}
  Consider $S = \{n \in \mathbb{N}: n > 1 \land n \text{ does not have a prime factorization}\}$. Then, we take the smallest element in this set, $k$. $k$ cannot be a prime, so there exists $a, b < k$ such that $k = ab$. However, $a$ and $b$ cannot be in $S$, since they are smaller than $k$. This means that $a = \prod {p_{a,k}}^{a_k}, b = \prod {p_{b, k}}^{b_k}$, meaning that their product is the product of primes. Therefore, $k$ cannot be in $S$, so $S$ has to be empty.
\end{proof}

\section{Theorem \ref{disjointorb}}
\begin{theorem*}
  Let $E$ a finite set, $x, y \in E$, $G$ a finite group. Then, 
  \[
    \Orb_x = \Orb_y \text{ ou } \Orb_x \cap \Orb_y = \emptyset
  \]
\end{theorem*}
\begin{proof}
  Assume that there exists an $s \in Orb_x : s \in Orb_y$. Then, there exist $g_x, g_y \in G$ such that
  \[
    s = g_x \cdot x = g_y \cdot y \Leftrightarrow x = g_x^{-1}g_y \cdot y = g \cdot y
  \]
  meaning that $x \in \Orb_y$, which in turn means $\Orb_x = \Orb_y$. Therefore, if there is an intersection between two orbits, then they must be the same. 
\end{proof}
  \input{chapters/appendixB.tex}
  \chapter{Jack gets pedantic about notation for like half a page}

Now. Professor Lachowska is a doctor in mathematics, and also potentially a good dozen of orders of magnitudes smarter than me. So this is by no means a way for me to claim I'm better than her. However. 

When you say that $\mathbb{R}[x]$ is the set of all polynomials of one variable over coefficients in $\mathbb{R}$, what you're doing in computer science terms is you're defining a type. An element of that set essentially follows the rules of that type. The way the type $\mathbb{R}[x]$ is typically notated is $\mathbb{R} \to \mathbb{R}$, and we define an element of that type as $f: \mathbb{R} \to \mathbb{R}$. 

$f(x)$ means the application of $f: \mathbb{R} \to \mathbb{R}$ to the number $x: \mathbb{R}$, and it is therefore a real value, $f(x): \mathbb{R}$. 

What Prof. Lachowska says is: ``Nah, I'll do my own thing''. $f(x)$ is the polynomial, and $f$ does not exist. This makes notation heavier and not standard. 

Why does she do that? I'm not sure. Does it matter? Not really. Am I bitter about it? Yes. 

And it's my textbook, so I get to write this page. 
\end{document}